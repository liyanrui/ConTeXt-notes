\chapter{列表}

若是帮领导起草并排版一份会议讲话稿,须知天下没有领导不偏爱含有列表的文章。大可以相信,\CONTEXT\ 列表绝对不会让领导失望。

\section{待办事项}

在准备用 ChatGPT 或 DeepSeek 给领导起草讲稿之前,先用 \CONTEXT\ 列表\index[itemize]{\type{itemize} 环境}安排一下今日待办事项:

\startexample
2023 年 3 月 22 日
\startitemize
\item 中午,晒十五分钟太阳
\item 晚上,看流浪地球 \Romannumerals{2}
\stopitemize
\stopexample
\example[option=TEX][todo-list]{待办事项}{\externalfigure[05/todo-list.pdf][width=.3\textwidth]}

现在你已经学会了列表的用法了,余下的事,大多是在设定它的样式。此外,你也学到了如何制作大写罗马数字。若需要制作小写罗马数字,只需将 \type{\Romannumerals} 换成 \type{\romannumerals}。

\section{无序号列表}

例 \in[todo-list] 已经展示了样式最为简单的无序号列表,列表项符号是实心圆点。该例中的列表实际上省略了列表项符号的设定,其完整形式为

\starttyping[option=TEX]
\startitemize[1]
\item 中午,晒十五分钟太阳
\item 晚上,看流浪地球 \Romannumerals{2}
\stopitemize
\stoptyping

将上述代码中的数字 \type{1} 换成 \type{2~9} 的任何一个数字,可更换序号样式。例如

\startexample
\startitemize[8]
\item 中午,晒十五分钟太阳
\item 晚上,看流浪地球 \Romannumerals{2}
\stopitemize
\stopexample
\example[option=TEX][]{改变列表项符号}{\externalfigure[05/todo-list-2.pdf][width=.3\textwidth]}

\section{有序号列表}

将无序号列表的符号参数换为 \type{n},便可得到有序号列表。例如

\startexample
\startitemize[n]
\item 中午,晒十五分钟太阳
\item 晚上,看流浪地球 \Romannumerals{2}
\stopitemize
\stopexample
\example[option=TEX][]{有序号列表}{\externalfigure[05/todo-list-3.pdf][width=.3\textwidth]}

有时,需要序号形式是带括号的数字,可像下面这样设定:

\startexample
\startitemize[n][left=(,right=),stopper=]
\item 中午,晒十五分钟太阳
\item 晚上,看流浪地球 \Romannumerals{2}
\stopitemize
\stopexample
\example[option=TEX][]{数字带括号的有序号列表}{\externalfigure[05/todo-list-4.pdf][width=.3\textwidth]}

\noindent 其中\boxquote{\type{stopper=}}是将参数 \type{stopper} 置空,从而消除列表项序号之后的西文句点\boxquote{\type{.}}。

若将列表项序号参数设定为 a,A,r,R,对应的序号形式分别为小写英文字母、大写英文字母、小写罗马数字、大写罗马数字。

\section{留白}

消除列表项之间的空白,只需

\startexample
2023 年 3 月 22 日
\startitemize[1,packed]
\item 中午,晒十五分钟太阳
\item 晚上,看流浪地球 \Romannumerals{2}
\stopitemize
\stopexample
\example[option=TEX][]{消除列表项之间的空白}{\externalfigure[05/todo-list-7.pdf][width=.3\textwidth]}

消除列表前后以及列表项之间的空白,只需

\startexample
2023 年 3 月 22 日
\startitemize[1,nowhite]
\item 中午,晒十五分钟太阳
\item 晚上,看流浪地球 \Romannumerals{2}
\stopitemize
\stopexample
\example[option=TEX][]{消除列表项之间的空白}{\externalfigure[05/todo-list-8.pdf][width=.3\textwidth]}

\section[itemcite]{引用}

在写论文时,通常要引用一些文献。这些文献通常是以列表的形式附在论文之后,在正文里以引用的方式自动呈现文献在列表中的序号。如此,当文献在列表中的次序有所调整时,正文中的引用所呈现的序号也会相应变化,而无需手工维护。基于列表的引用机制\index[liebxyy]{列表项引用}可以完成这类任务。

\startexample
\title{论文}

正文引用了文献[\in[胡说]]和文献[\in[八道]]……

\subject{参考文献}
\startitemize[n,joinedup]
             [left={[},right={]},
              distance=.5em, stopper=]
\item[胡说] 这是胡说的论文。
\item[八道] 这是八道的论文。
\stopitemize
\stopexample
\example[option=TEX][reflist]{参考文献列表及引用}{\externalfigure[05/reference.pdf][width=.425\textwidth]}

上例中,列表的 \type{distance} 参数用于设定列表项的序号与内容的间距,我将其设定为半个字宽。\tex{item} 命令后的方括号中的内容是引用名。在正文里,使用 \tex{in} 命令可以通过列表项的引用名获得其序号。

不过,这种基于列表的参考文献机制只适用于文献数量不多的情况。倘若有大量的参考文献数据,而文章仅引用其中一部分,对于此类任务,\CONTEXT\ 提供了更为专业的参考文献数据管理和引用机制,该机制的基本用法,见 \in[bibtex] 章。

\section[drawing-sym]{画符}

上文所述的列表样式用于常规文档的排版应该是足够用的,但现实中还有许多非常规文档,例如幻灯片、海报和杂志等,这类文档需要排版元素具有美学特征,而非朴素的科技文档特征。\CONTEXT\ 允许我们使用 MetaPost 语言画出我们想要的列表符号。

MetaPost 是一种语法简洁但功能丰富的矢量绘图编程语言。\CONTEXT\ 早已内嵌了该语言的解释器,我们可以直接在 \CONTEXT\ 源文件里编写 MetaPost 代码,然后交由 \CONTEXT\ 编译器生成嵌入在排版结果中的图形。本文档的第 \in[metapost] 章讲述了 MetaPost 的基本语法,故而若下文代码超出你的理解范围,可先行阅读该章内容,也先保留不解,待后续读至该章时自然获解。此外,文档 \cite[mp-necessity,mp-ethos,mp-crow] 亦可供参考。

由于 \CONTEXT\ 提供的列表项符号皆为黑色图案,现在我们用 MetaPost 语言在 \type{uniqueMPgraphic} 环境\index[uniqueMPgraphic]{\type{uniqueMPgraphic} 环境}里画带阴影且用复合色填充的正方形,见下例,准备用它作为列表项符号。

\startMP
\startuniqueMPgraphic{带阴影的正方形}
numeric u; path p;
u := BodyFontSize; 
p := fullsquare scaled .8u;
fill p shifted (.2u, -.2u) withcolor lightgray;
fill p withcolor .5[blue, white];
\stopuniqueMPgraphic
\stopMP

上述 \type{uniqueMPgraphic} 环境中的 MetaPost 代码可以作为插图使用,见下例。

\startexample
这个\,\uniqueMPgraphic{带阴影的正方形}\,是
MetaPost 图形。
\stopexample
\example[option=TEX][square]{带阴影的正方形图案}{\externalfigure[05/square.pdf][width=.425\textwidth]}

注意,上例源码中的\boxquote{\tex{,}}是 \TEX\ 命令,用于构造 $\frac{1}{6}$ 字宽的间距。顺便记着,\boxquote{\tex{;}}可构造 $\frac{5}{18}$ 字宽的间距,而一个字宽的空格可以用 \tex{quad} 命令构造\index[kongge]{空格}。有时在遇到普通空格过宽,或者空格被 \CONTEXT\ 编译器吞噬时,可以用这些 \TEX\ 间距命令构造空格。关于 \CONTEXT\ 里的空格,你需要知道,汉字之间的空格会被 \CONTEXT\ 编译器吞噬掉,而汉字与西文字符之间如果存在多个空格,\CONTEXT\ 编译器只保留 1 个。

现在,请认真观察上例中的正方形图案,应该能发现它的位置与同一行的其他文字相比有些上浮,这是因为图形对象缺乏字符基线所致。字符基线是西文字体特有的概念,还记得小时候学英文时用的四线三格本子吗?那 4 条线从上往下数的第 3 条线便是西文字符的基线\index{baseline,基线}。这些西文字符的基线连接起来,便构成一行文字的基线,所有的汉字都在基线上面,有几个英文字符,如 \type{g}、\type{p}、\type{y} 等,它们的有一部分沉于基线之下。于是,一行文字里,字符的最大高度是由基线之上的部分加上基线之下的部分,这两个部分的尺寸比例是 $0.72:0.28$。上述示例创建的正方形图形,它的高度是最大字符高度,但是在排版时,它像汉字那样位于基线之上了,故而显得有些上浮。要解决这个问题,需要将图形置于 \tex{hbox} 命令\index[hbox]{\tex{hbox}}创建的水平盒子里,再用 \tex{lower} 命令\index[lower]{\tex{lower}}让这个盒子下沉 0.2 倍的 \tex{bodyfontsize},见下例。

\startTEX
这个\,\lower.2\bodyfontsize\hbox{\uniqueMPgraphic{带阴影的正方形}}\,是 MetaPost 图形。
\stopTEX
\placeExample[here][square]{带阴影的正方形图案}{\externalfigure[05/square-2.pdf][width=.425\textwidth]}

\noindent 为什么要下沉 0.2 倍的正文字号呢?原因是,作为阴影的正方形相对于淡蓝色的正方形向下偏移了 0.25 的正文字号。如果让包含了这个 MetaPost 图形的盒子下沉这个距离,刚好能让淡蓝色的正方形底线位于基线上。

用 \tex{definesymbol} 命令\index[definesymbol]{\tex{definesymbol}}可将这个正方形图案定义为列表项符号,由于 \CONTEXT\ 的 9 种列表项符号的序号是从 1 至 9,故而我们自己画的符号可从序号 10 开始,即

\startTEX
\definesymbol[10][{\lower.2\bodyfontsize\hbox{\uniqueMPgraphic{带阴影的正方形}}}]
\stopTEX

\noindent 之后便可在列表环境里使用这个符号,见下例。

\startexample
\startitemize[10,nowhite]
\item 中午,晒十五分钟太阳
\item 晚上,看流浪地球 \Romannumerals{2}
\stopitemize
\stopexample
\example[option=TEX][squaresym]{列表项的画符}{\externalfigure[05/todo-list-a.pdf][width=.3\textwidth]}

\section[lua]{带圈的数字}

对于有序列表,有些文档格式要求用带圆圈的数字,\CONTEXT\ 没有提供这种风格的列表项序号。不过,即使你并未明白上一节的列表项画符的一些技术细节,应该也有一些这样的信心——\CONTEXT\ 未提供的,我们总有办法创造出来。

\CONTEXT\ 支持 Unicode 编码的文字,而汉字字体通常提供了带圈的数字,从 1 至 9,可以用 \tex{char} 命令\index[char]{\tex{char}}获得它们,例如

\startTEX
\char"2460,\char"2461,\cdots,\char"2468
\stopTEX

\noindent 结果为 \char"2460,\char"2461,\cdots,\char"2468。

在列表环境里,可以用 \tex{sym}\index[sym]{\tex{sym}} 代替 \tex{item},直接指定我们想要的序号。例如

\startexample
\startitemize
\sym{\char"2460} 中午,晒十五分钟太阳
\sym{\char"2461} 晚上,看流浪地球 \Romannumerals{2}
\stopitemize
\stopexample
\example[option=TEX][circled]{带圈的序号}{\externalfigure[05/todo-list-b.pdf][width=.3\textwidth]}

这样虽然能解决问题,但颇为不雅,亦即 Donald Knuth 所说的,快速而脏的东西。不过,要想得到高雅的结果,需要耗费一些心力,甚至需要一些编程。不过,实际上当你开始用 \CONTEXT\ 排版第一个示例时,你已经在编程了。\TEX\ 是一门排版编程语言,你见过的任何一个 \CONTEXT\ 命令,本质上都是一小段甚至上千行的 \TEX\ 程序。

\TEX\ 编程对于大多数人而言,难以掌握,但是从 2000 年代末开始,\CONTEXT\ 开发者将更易掌握的 Lua 语言引入 \TEX\ 世界,使得 \TEX\ 编程变得容易许多,当然这需要你对 Lua 语言有所了解,文档 \cite[basic-lua] 是为 \CONTEXT\ 用户所写的 Lua 语言介绍。

下文内容,也许会让你觉得难以理解,可以跳过去,也可以可先观其形,待你对 \CONTEXT\ 理解得更为深入时,再解其意。

有序号列表的序号可用 \tex{defineconversion} 命令\index[defineconversion]{\tex{defineconversion}}定义。例如

\startTEX
\defineconversion[带圈数字][\CircledNumber]

\startitemize[带圈数字]
\item ... ... ...
\item ... ... ...
\stopitemize
\stopTEX

\noindent 关键在于 \tex{CircledNumber} 命令如何定义,该命令的形式如下

\startTEX
\def\CircledNumber#1{... 定义 ...}
\stopTEX

\noindent 其中 \type{#1} 表示 \tex{CircleNumber} 接受 1 个参数,即列表项的序号。由于 \tex{CircledNumber} 是由 \type{itemize} 环境触发的,且后者每次触发它时,就将当前列表项的序号传给它。于是,我们在实现 \tex{CircledNumber} 的定义时,所考虑的问题仅仅是如何将列表项的序号映射为带圆圈的数字,这个过程过去只能用 \TEX\ 语言实现,现在可以用 Lua 语言实现。不过,我们最好事先做一下热身运动,实现一种 \CONTEXT\ 已经实现了的序号,见下例。

\startexample
\def\FooNumber#1{【#1】}
\defineconversion[Foo][\FooNumber]

\startitemize[Foo][distance=1em, stopper=]
\item 中午,晒十五分钟太阳
\item 晚上,看流浪地球 \Romannumerals{2}
\stopitemize
\stopexample
\example[option=TEX][foonumber]{汉字方括号序号列表}{\externalfigure[05/todo-list-c.pdf][width=.3\textwidth]}

上例中的列表实际上等效于

\startTEX
\startitemize[n, nowhite][left=【, right=】, distance=1em, stopper=]
... ... ...
\stopitemize
\stopTEX

\noindent 不过基于该示例,你应该能够清晰地观察到 \tex{FooNumber} 所接受的参数是什么以及我们如何使用它的,只要将这个参数传递给一个 Lua 语言编写的函数,由后者将其映射为带圆圈的数字,则我们的问题便得以解决。现在,先写出这个 Lua 函数。\CONTEXT\ 允许我们在 \type{luacode} 环境里编写 Lua 代码\index[luacode]{\type{luacode} 环境},见以下代码:

\startLUA
\startluacode
function to_circled_number(x)
  context('\\char"' .. tostring(2460 + x - 1))
end
\stopluacode
\stopLUA

\CONTEXT\ 编译器并非在同一个空间里执行 \CONTEXT\ 排版代码和嵌入的 Lua 代码,这两种代码各自有一个运行空间,毕竟它们是不同的编程语言。Lua 空间里的数据需要以字符串对象的形式通过 \type{context} 函数\index[contextfunc]{\type{context} 函数}传回 \CONTEXT\ 空间。

上述函数 \type{to_circled_number} 所作的工作是,将外界传入的数值 \type{x} 减 1,再与 2460 相加,然后将所得结果用 \type{tostring} 转换为字符串,在其前面加上 \tex{char"} 前缀,最后由 \type{context} 函数将结果传回 \CONTEXT\ 空间。例如,假设 \type{x} 是 \type{3},经过 \type{to_circled_number} 的一番处理,它变成了 Lua 字符串 \type[escape=yes]{'\\char"2462'},该结果由 \type{context} 函数传回 \CONTEXT\ 空间。

同理,在 \CONTEXT\ 空间调用 Lua 空间里的一个函数,也需要一个通道,该通道由 \tex{ctxlua} 命令\index[ctxlua]{\tex{ctxlua}}实现,基于这个命令便可定义 \tex{CircledNumber} 命令,即

\startTEX
\def\CircledNumber#1{\ctxlua{to_circled_number(#1)}}
\stopTEX

\noindent 于是,用带圆圈数字作为列表序号的任务便如此优雅地完成了。现在不妨登向高处,俯瞰整个方案。实际上并不难,对吗?

\startexample
\startluacode
function to_circled_number(x)
  context('\\char"' .. tostring(2460 + x - 1))
end
\stopluacode

\def\CircledNumber#1{\ctxlua{to_circled_number(#1)}}
\defineconversion[Foo][\FooNumber]

\startitemize[CircledNumber][stopper=]
\item 中午,晒十五分钟太阳
\item 晚上,看流浪地球 \Romannumerals{2}
\stopitemize
\stopexample
\example[option=TEX][good-circled]{Lua 与 \TEX\ 的结合}{\externalfigure[05/todo-list-b.pdf][width=.3\textwidth]}

\subject{结语}

本章的最后两节,可能会让你有一些压迫感。也许你原以为 \CONTEXT\ 已经为你提供好了一切命令,无论你想要什么样的排版效果,只需要找到相应的命令便可达成目的。这当然是不可能的,否则 \CONTEXT\ 开发者就成了上帝,他们知晓你的一切心意。上帝是不存在的,但这个世界是可编程的。
