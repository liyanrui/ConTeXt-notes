\chapter{参考文献}

稍微严肃一些的文章,往往会附上一些与文章内容密切相关的参考文献。科研论文更是如此,牛顿都说过自己是站在巨人的肩膀上做事情的。参考文献便是巨人。任何一个 \TeX\ 系统皆不敢对支持参考文献排版这一事宜掉以轻心,否则 \TeX\ 无以为科技文献排版软件之先驱和典范。

\section{\BibTeX}

大多数 \TeX\ 系统在排版参考文献时,皆需依赖 bibtex 程序。由 bibtex 基于文献数据文件生成参考文献列表信息,然后交由 \TeX\ 进行排版,这一过程通常是自动进行的,并不需要用户了解和干预。\ConTeXt\ 现在不需要依赖 bibtex 程序,它已自身内部实现了 bibtex 的全部功能。本文档在谈及 \BibTeX\ 时,主要指其文献数据格式。

\BibTeX\ 文献数据文件是纯文本文件,其中包含着论文、专著和手册等文献数据。例如,一篇期刊论文,其数据格式为

\starttyping
@article{knuth-1984-literate,
  title={Literate programming},
  author={Knuth, Donald Ervin},
  journal={The computer journal},
  volume={27},
  number={2},
  pages={97--111},
  year={1984},
  publisher={Oxford University Press}
}
\stoptyping

\noindent 再例如一本专著,其数据格式为

\starttyping
@book{knuth-1986-texbook,
  title={The texbook},
  author={Knuth, Donald Ervin and Bibby, Duane},
  volume={1993},
  year={1986},
  publisher={Addison-Wesley Reading, MA}
}
\stoptyping

现在,为了学习 \ConTeXt\ 的参考文献排版功能,可将上述文献数据保存为一份文本文件,例如 foo.bib,将该文件作为文献数据文件。

在 foo.bib 相同目录下,建立 \ConTeXt\ 源文件,例如 foo.tex,其内容为

\starttyping[option=TEX]
\usebtxdataset[foo.bib] % 加载 foo.bib
\starttext
Knuth 基于文学编程\cite[knuth-1984-literate]技术开发了
\TeX\ 系统\cite[knuth-1986-texbook]。
\blank
\placelistofpublications
\stoptext
\stoptyping
\noindent 编译 foo.tex,可得以下结果:
\blank[line]
\midaligned{\externalfigure[06/bibtex-example-01.pdf]}
\blank[line]

\section{文献列表样式}

上一节示例中的文献列表是 \ConTeXt\ 默认样式,除此之外,还有其他三种预定义的文献列表样式:apa,aps 和 chicago,可在 \type{\placelistofpublications} 之前使用 \type{\usebtxdefinitions} 进行切换。例如使用 aps 样式,

\starttyping[option=TEX]
... ... ...
\usebtxdefinitions[aps]
\placelistofpublications
\stoptyping
\blank[line]
\midaligned{\externalfigure[06/bibtex-example-02.pdf]}
\blank[line]

对于预定义样式,可以进行调整。例如,缩小文献序号后的空白间距,

\starttyping[option=TEX]
\setupbtxlist[apa][distance=.5em,width=fit]
\stoptyping
\blank[line]
\midaligned{\externalfigure[06/bibtex-example-03.pdf]}

\section{自定义文献列表样式}

也许 \ConTeXt\ 提供的参考文献列表样式并不符合你的需求,因此你会忍不住想自己定义一种样式。对此,我的看法是,文献列表样式当由期刊编辑部或专著出版商负责定义,个人无需如此刻意。不过,倘若你正是此类机构工作人员,需要为 \ConTeXt\ 定义符合自己单位要求的文献样式,下文仅能为你提供一个简单的示例,希望对你有所帮助。

现在,假设我们要定义一种名字叫作 foo 的文献列表样式。首先,按照 \ConTeXt\ 的文件命名约定,建立两份文件,一份是 publ-imp-foo.lua,一份是 publ-imp-foo.tex。在 publ-imp-foo.lua 里写入以下内容:

\starttyping[option=LUA]
local specification = {
    name = "foo",
    categories = {},
}
local categories = specification.categories
categories.article = {
    required = {"author"},
    optional = {
        "year",
        "title", "journal", "volume", "number", "pages"
    }
}
return specification
\stoptyping

\noindent 上述代码是 Lua 代码,定义了一个表 \type{specification},设定了期刊论文必要和可选元素。

在 publ-imp-foo.tex 文件中写入以下内容

\starttyping[option=TEX]
\definebtx[foo][specification=foo]
\definebtxrendering[foo][specification=foo]
\stoptyping

\noindent 上述代码定义了新的文献数据格式 \type{foo} 及其样式。注意,这两处定义皆需要让 \type{specification} 指向 publ-imp-foo.lua 文件中的名字为 \type{foo} 的 \type{specification} 表。

继而,为期刊论文设置文献列表样式,在 publ-imp-foo.tex 文件中添加以下内容:

\starttyping[option=TEX]
\startsetups btx:foo:list:article
  \texdefinition{btx:foo:author}
  \texdefinition{btx:foo:title}
  \texdefinition{btx:foo:journal}
  \btxperiod % 表示西文句号
\stopsetups
\stoptyping

\noindent 上述代码定义了期刊论文的构成,并以西文句号作为每一项期刊论文数据的结尾标点。

在 publ-imp-foo.tex 文件中添加期刊论文各元素的构成:

\starttyping[option=TEX]
\starttexdefinition btx:foo:author
  \btxdoifelse{author}{\color[blue]{\btxflush{author}}}{No name}
  \btxperiod\btxspace
\stoptexdefinition
\stoptyping

\starttyping[option=TEX]
\starttexdefinition btx:foo:title
  \btxdoifelse{title}
              {\color[red]{\btxflush{title}}\btxperiod\btxspace}
              {No title}
  \btxperiod\btxspace
\stoptexdefinition
\stoptyping

\starttyping[option=TEX]
\starttexdefinition btx:foo:journal
  \btxdoif {journal} {
    \color[magenta]{\btxflush{journal}}\btxcomma\btxspace
    \btxflush{year},\nospace
    \btxdoifelse {volume} {
      \btxspace
      \btxflush{volume}
      \btxdoif {number} {
        \ignorespaces
        \btxleftparenthesis
        \btxflush{number}
        \btxrightparenthesis
      }
    }{
      \btxdoif {number} {
        \btxlabeltext{default:number}\btxspace\btxflush{number}
      }
    }
    \btxdoif {pages} {\nospace\btxcolon\btxspace\btxflush{pages}}
  }
\stoptexdefinition
\stoptyping

\noindent 上述代码看似复杂,但逻辑较为简单,其中 \type{\btxdoif} 和 \type{btxdoifelse} 用于判断 \ConTeXt\ 对 \BibTeX\ 数据格式的解析结果中是否包含某项并作相应处理,亦即 \CONTEXT\ 的参考文献功能在一定程度上具备可编程特性。\type{\btxflush} 可将 \ConTeXt\ 从 \BibTeX\ 数据格式中解析出的信息输出到排版结果中。\type{\btxcomma} 为西文逗号,\type{\btxspace} 为西文空格。\type{\ignorespaces} 用于忽略空格。\type{\btxleftparenthesis} 和 \type{\btxrightparenthesis} 分别为「\type{(}」和「\type{)}」。

在 publ-imp-foo.lua 和 publ-imp-foo.tex 文件所在目录下,建立测试文件 foo.tex,其内容为

\starttyping[option=TEX]
\usebtxdataset[foo.bib] % 加载 foo.bib
\starttext
Knuth 发明了文学编程语言 WEB\cite[knuth-1984-literate]。
\blank
\usebtxdefinitions[foo] % 使用上文定义的参考文献列表样式 foo
\placelistofpublications
\stoptext
\stoptyping

用于验证上述定义参考文献类表样式 foo 是否可用。结果为

\blank[line]
\midaligned{\externalfigure[06/bibtex-example-04.pdf]}

\section{廉价方案}

倘若你仅仅是在写一份笔记或手册,若半个小时依然未能学会如何为 \ConTeXt\ 定义参考文献样式,我建议随便选一个 \ConTeXt\ 预定义的样式使用。待文章最终定稿后,手工用 \ConTeXt\ 列表将参考文献列表制作成你喜欢的样式,与创作正文内容相比,所费时间通常可忽略不计。例如

\startsample
\def\Cite[#1]{[\in[#1]]}
著名废话学家李某人\Cite[limouren]说过,正经人谁会看参考文献啊!
\startitemize[n,joinedup][left={[},right={]},stopper=]
\item[limouren] 李某人.如何说正确的废话[J].废话学报, 2023, 1(1): 1-10000.
\stopitemize
\stopsample
\typesample
\startblueframedtext
\getsample
\stopblueframedtext

\section{小结}

Hans Hagen 为 \ConTeXt\ 对参考文献的支持专门写了一份约 100 页的手册,见

\starttyping
$ mtxrun --script base --search mkiv-publications.pdf
\stoptyping

\noindent 若有兴致,不妨一睹。