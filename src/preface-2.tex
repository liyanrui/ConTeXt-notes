\title{自序之二}

两年过去了,这份文档该更新了。漫长的时间里,偶尔还能继续以前的工作,会显得人颇为执着。实际上并非如此,这两年里,我几乎已经将这份文档忘记了。

那一年,我已忘记是哪一年了,奶海寄给我一本《\LATEX\ 入门》,他写的,很厚,也很全面,在我看来,是中文 \LATEX\ 书里迄今为止最好的一本。我当时正在百度贴吧里玩刀,心里想的都是蝴蝶 581,蜘蛛 C81,Esee 3……已将 \TEX\ 忘记许多年了,甚至连 CTeX 论坛何时倒闭的都不知情,而我还一直忝居某个版块的版主。如果我不讲这个故事,那么在熟悉我的人看来,我十七年如一日喜欢着 \CONTEXT。

这一次也是如此,我可以用十多天的时间,修缮这份文档并补充一些我感兴趣的主题,然后我会继续忘记它。

我发现,为某个事物写文档,最大的好处就是写好之后可以放心忘记它。当我收到奶海寄来的书时,我是忘记了 \CONTEXT\ 的很多知识,但这种忘记是此心难安的。放心式的忘记,第一次感受到它,是在上一次完成这份文档之后。尽管在此之前,我在自己的几个博客上写过一些笔记,但是写博客终归是太过于随意且虚荣,并不能支撑这种放心式的忘记。

当我真正理解了自己的这个想法后,我可以正面回答 \TEX\ 爱好者们经常要面对的一个问题,即 \TEX\ 比 MS Word 这种大众化意义上的排版软件,比 Markdown 这样的轻量级标记语言……好在何处呢?答案是,在不预设立场的前提下,分别用三者为同一事物编写文档,之后忘记这份文档的放心程度是 $\text{\TEX} > \text{MS Word} > \text{Markdown}$。

做一件事情,有许多工具可选,而你选择的工具,用起来最难但是能得到好的结果,这意味着你对这件事极为重视,无形之中拓展了对它的思考广度和深度。这就是 \TEX\ 的长处,它能帮助你创造这样的机会。

Donald Knuth 是为了排版他所写的《计算机编程艺术》这套书而创造了 \TEX,他从中得到的放心忘记的机会应该远大于我辈 \TEX\ 用户。为什么要追求放心式的忘记呢?为了尽量减少杂念。之后的几年里,如果我又重新玩刀或者别的什么,忽然有一天,我收到了奶海寄来的《\LATEX\ 入门》第 2 版,我心里不会再收到由愧意和嫉妒交织成的复杂情感,并在它的驱动下后责问自己,你还会多少 \CONTEXT\ 呢?

如果你觉得 \TEX\ 太难——无论是 \LATEX\ 还是 \CONTEXT,学习成本过高,故而觉得用它编写文档,得不偿失。我的建议是,你真正应该考虑的是,你所写的文档,你是否需要放心忘记它。事实上,大多数文档连忘记都是不配提的,更勿言放心。

如果将 \TEX\ 生成的 PDF 打印到纸上,你不仅可以放心忘记它,甚至无需担心整个计算机世界的覆灭。

\vfill
\startalignment[flushright,broad]
2025 年 8 月写于淄博蜗居
\stopalignment
